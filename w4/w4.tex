
\documentclass[a4paper,12pt]{article}

\usepackage{cmap}					% поиск в PDF
\usepackage[T2A]{fontenc}			% кодировка
\usepackage[utf8]{inputenc}			% кодировка исходного текста
\usepackage[english,russian]{babel}	% локализация и переносы

\documentclass{report}
\usepackage[utf8]{inputenc}
\usepackage[english,russian]{babel}
\usepackage[a4paper,top=1cm,bottom=2cm,left=2cm,right=2cm,marginparwidth=1.75cm]{geometry}
\usepackage{amsfonts}
\usepackage{amsmath}
\usepackage{amsthm}
\usepackage{amssymb}
\usepackage{graphicx}
\usepackage{hyperref}
\usepackage{mathtools}
\usepackage{geometry}
\usepackage{enumitem}
\usepackage[usenames,dvipsnames]{xcolor}
\usepackage{tikz}
\thispagestyle{empty}
\usetikzlibrary{
  graphs,
  graphs.standard
}
\usepackage{etoolbox}
\usepackage{url}    % package url to prevent horrible linebreaks
\usepackage{tikz}
\usetikzlibrary{positioning,chains,fit,shapes,calc}

\usepackage{tikz}
\usetikzlibrary{graphs,graphs.standard}


\setlength\parindent{0pt}

\numberwithin{equation}{section}
\theoremstyle{plain}
\newtheorem{theorem}{Теорема}[section]
\newtheorem{lemma}[theorem]{Лемма}
\newtheorem{corollary}[theorem]{Corollary}
\newtheorem{proposition}[theorem]{Предложение}
\newtheorem{conjecture}[theorem]{Гипотеза}
\newtheorem{criterion}[theorem]{Criterion}
\newtheorem{algorithm}[theorem]{Algorithm}
\theoremstyle{definition}
\newtheorem{definition}[theorem]{Определение}
\newtheorem{condition}[theorem]{Condition}
\newtheorem{problem}[theorem]{Problem}
\newtheorem{example}[theorem]{Пример}
\newtheorem{exercise}{Exercise}[section]
\newtheorem{obs}{Observation}
\theoremstyle{remark}
\newtheorem{remark}[theorem]{Remark}
\newtheorem{note}[theorem]{Note}
\newtheorem{notation}[theorem]{Notation}
\newtheorem{claim}[theorem]{Claim}
\newtheorem{summary}[theorem]{Резюме}
\newtheorem{acknowledgment}[theorem]{Acknowledgment}
\newtheorem{case[theorem]}{Case}
\newtheorem{conclusion}[theorem]{Conclusion}

\newtheorem*{theorem*}{\bf Теорема}
\newtheorem*{lemma*}{\bf Лемма}
\newtheorem*{definition*}{\bf Определение}
\newtheorem*{example*}{\bf Пример}
\newtheorem*{remark*}{\bf Замечание}
\newtheorem*{hypothesis*}{\bf Гипотеза}


\hypersetup{
  colorlinks   = true,    % Colours links instead of ugly boxes
  urlcolor     = blue,    % Colour for external hyperlinks
  linkcolor    = blue,    % Colour of internal links
  citecolor    = red      % Colour of citations
}

\def\N{\mathbb {N}}
\def\Z{\mathbb {Z}}
\def\F{\mathbb {F}}
\def\d{\delta}
\def\Gr{{\mathbf G}}
\def\E{\mathbb {E}}
\def\eps{\varepsilon}
\def\Sym{{\rm Sym}}
\def\FF{\widehat}
\def\C{\mathbb{C}}

\DeclareMathOperator*\lowlim{\underline{lim}}
\DeclareMathOperator*\uplim{\overline{lim}}

\usepackage{scalerel,stackengine}
\stackMath
\newcommand\reallywidehat[1]{%
\savestack{\tmpbox}{\stretchto{%
  \scaleto{%
    \scalerel*[\widthof{\ensuremath{#1}}]{\kern.1pt\mathchar"0362\kern.1pt}%
    {\rule{0ex}{\textheight}}%WIDTH-LIMITED CIRCUMFLEX
  }{\textheight}% 
}{2.4ex}}%
\stackon[-6.9pt]{#1}{\tmpbox}%
}
\parskip 1ex


%\author{}
%\title{1.1 Наш первый документ}
%\date{\today}

\begin{document} % Конец преамбулы, начало текста.
%\maketitle

Бойко Александр
\newline
\newline
Неделя 4, Задача 1 \newline
Условие. Пусть $A \subseteq \{ 1,2,\dots, N \}$ --- произвольное множество без
нетривиальных аддитивных четверок, то есть без решений уравнения $a+b = c+d$, где  $a,b,c,d \in A$ --- различные.
Доказать, что для некоторой константы $C>0$ выполнено $|A| < C\sqrt{N}$ (то есть множество $A$ не может быть слишком большим).
\begin{proof}
Для всех неупорядоченных пар $(a_1, a_2) \in A$ выпишем значения $a_1 + a_2$. \newline
Все выписанные значения различны (по условию), при этом их $C_{|A|}^2$ штук, кроме того они от 2 до $2N$ \newline
Значит, $2N > 2N-1 \geq C_{|A|}^2$ \newline \newline
$2N > \frac{|A|(|A|-1)}{2} \geq \frac{|A|^2}{4}$ \newline \newline
$|A| < \sqrt{8N} \Rightarrow$ искомая константа $\sqrt{8}$ 
\end{proof} \newline \newline
Задача 2. \newline
Условие. Доказать, что в абелевой группе выполнено $|nA| \le \binom{|A|+n-1}{|A|-1}$.
\begin{proof}
В доказательстве будем пользоваться обозначением биномиального коэффициента $C_n^k$. \newline
Обозначим элементы мн-ва $A$, как $a_1, a_2, \ldots, a_k$ \newline
Мн-во $nA$ состоит из сумм такого вида: $a_{i_1} + a_{i_2} + \ldots + a_{i_n}$ \newline
поскольку группа абелева, то значение суммы не зависит от порядка $\Rightarrow$ можно оценить $|nA|$ сверху, как количество мульти-множеств $i_1, \ldots, i_n$. \newline
Выберем из $k$ элементов $n$ с возвращением. Все мульти-множетсва это те и ровно те множества, которые получаются по схеме выбора $n$ элементов из $k$ с возвращением, таким образом мульти-множеств $C_{n+k-1}^{k-1} = C_{n+|A|-1}^{|A|-1}$ штук
\end{proof}

Задача 4. \newline
Условие. Доказать, что для всякого натурального $k$ выполнено
$$
|A-2^{k} \cdot A| \le |A| \cdot \left( \frac{|A-2\cdot A|}{|A|} \right)^k \,.
$$
\begin{proof}
$$|A - 4 \cdot A| \leq \frac{|A-2 \cdot A| \cdot |2 \cdot A - 4 \cdot A|}{|2 \cdot A|} = \frac{|A-2\cdot A| \cdot |A-2\cdot A|}{|A|}$$
По индукции: \newline
База: $k=1$ \newline
$k \rightarrow k+1$ \newline
Ruzza's inequality \newline
$A=A$ \newline
$B=2^{k+1} \cdot A$ \newline
$C = 2^k \cdot A$ \newline
$$|A-2^{k+1}\cdot A| \leq \frac{|A-2^k\cdot A|\cdot |2^k\cdot A-2^{k+1}\cdot A|}{|2\cdot A|} = \frac{|A-2^k\cdot A|\cdot |A-2\cdot A|}{|A|} \leq$$ \newline $$\leq \frac{|A-2\cdot A|}{|A|} \cdot |A| \cdot (\frac{|A-2\cdot A|}{|A|})^k = |A|\cdot (\frac{|A-2\cdot A|}{|A|})^{k+1}$$
\end{proof}


\end{document} % Конец текста.

