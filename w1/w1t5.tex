
\documentclass[a4paper,12pt]{article}

\usepackage{cmap}					% поиск в PDF
\usepackage[T2A]{fontenc}			% кодировка
\usepackage[utf8]{inputenc}			% кодировка исходного текста
\usepackage[english,russian]{babel}	% локализация и переносы

\documentclass{report}
\usepackage[utf8]{inputenc}
\usepackage[english,russian]{babel}
\usepackage[a4paper,top=1cm,bottom=2cm,left=2cm,right=2cm,marginparwidth=1.75cm]{geometry}
\usepackage{amsfonts}
\usepackage{amsmath}
\usepackage{amsthm}
\usepackage{amssymb}
\usepackage{graphicx}
\usepackage{hyperref}
\usepackage{mathtools}
\usepackage{geometry}
\usepackage{enumitem}
\usepackage{etoolbox}
\usepackage{url}    % package url to prevent horrible linebreaks

\setlength\parindent{0pt}

\numberwithin{equation}{section}
\theoremstyle{plain}
\newtheorem{theorem}{Теорема}[section]
\newtheorem{lemma}[theorem]{Лемма}
\newtheorem{corollary}[theorem]{Corollary}
\newtheorem{proposition}[theorem]{Предложение}
\newtheorem{conjecture}[theorem]{Гипотеза}
\newtheorem{criterion}[theorem]{Criterion}
\newtheorem{algorithm}[theorem]{Algorithm}
\theoremstyle{definition}
\newtheorem{definition}[theorem]{Определение}
\newtheorem{condition}[theorem]{Condition}
\newtheorem{problem}[theorem]{Problem}
\newtheorem{example}[theorem]{Пример}
\newtheorem{exercise}{Exercise}[section]
\newtheorem{obs}{Observation}
\theoremstyle{remark}
\newtheorem{remark}[theorem]{Remark}
\newtheorem{note}[theorem]{Note}
\newtheorem{notation}[theorem]{Notation}
\newtheorem{claim}[theorem]{Claim}
\newtheorem{summary}[theorem]{Резюме}
\newtheorem{acknowledgment}[theorem]{Acknowledgment}
\newtheorem{case[theorem]}{Case}
\newtheorem{conclusion}[theorem]{Conclusion}

\newtheorem*{theorem*}{\bf Теорема}
\newtheorem*{lemma*}{\bf Лемма}
\newtheorem*{definition*}{\bf Определение}
\newtheorem*{example*}{\bf Пример}
\newtheorem*{remark*}{\bf Замечание}
\newtheorem*{hypothesis*}{\bf Гипотеза}


\hypersetup{
  colorlinks   = true,    % Colours links instead of ugly boxes
  urlcolor     = blue,    % Colour for external hyperlinks
  linkcolor    = blue,    % Colour of internal links
  citecolor    = red      % Colour of citations
}

\def\N{\mathbb {N}}
\def\Z{\mathbb {Z}}
\def\F{\mathbb {F}}
\def\d{\delta}
\def\Gr{{\mathbf G}}
\def\E{\mathbb {E}}
\def\eps{\varepsilon}
\def\Sym{{\rm Sym}}
\def\FF{\widehat}
\def\C{\mathbb{C}}

\DeclareMathOperator*\lowlim{\underline{lim}}
\DeclareMathOperator*\uplim{\overline{lim}}

\usepackage{scalerel,stackengine}
\stackMath
\newcommand\reallywidehat[1]{%
\savestack{\tmpbox}{\stretchto{%
  \scaleto{%
    \scalerel*[\widthof{\ensuremath{#1}}]{\kern.1pt\mathchar"0362\kern.1pt}%
    {\rule{0ex}{\textheight}}%WIDTH-LIMITED CIRCUMFLEX
  }{\textheight}% 
}{2.4ex}}%
\stackon[-6.9pt]{#1}{\tmpbox}%
}
\parskip 1ex


%\author{}
%\title{1.1 Наш первый документ}
%\date{\today}

\begin{document} % Конец преамбулы, начало текста.
%\maketitle

Бойко Александр
\newline
\newline
Неделя 1, Задача 5 \newline
Условие:
Доказать неравенство полуаддитивности для верхней плотности Банаха :
$$
D^* \left( \bigcup_{j=1}^k C_j \right) \le \sum_{j=1}^k D^* (C_j) \,.
$$

\begin{proof}
Можно заметить, что для любого $n$ и $U_1, \ldots, U_k$ выполнено неравенство \newline
$U_1, U_2, \ldots, U_k \subset \{1, \ldots,n\}$ \newline

$\frac{|U_1 \cup U_2 \cup \ldots U_k|}{n} \leq \frac{|U_1|}{n} + \frac{|U_2|}{n} + \ldots + \frac{|U_k|}{n}$ т.к Мощность объеденения не больше суммы мощностей \newline

Значит, если 
$U_i := C_i \cap \{1, \ldots, n\}$ \newline
$C_1 \cap \{1, \ldots, n\}) \cup (C_2 \cap \{1, \ldots, n\}) \cup \ldots \cup (C_k \cap \{1,\ldots, n\}) = (C_1 \cup C_2 \cup \ldots \cup C_k) \cap \{1, \ldots, n\}$

$x_n = \frac{|(C_1 \cup C_2 \cup \ldots \cup C_k) \cap \{1, \ldots, n\}|}{n}$ \newline

$y_{i,n} = \frac{|C_i \cap \{1,\ldots,n\}|}{n}$ \newline

Для любого $n$ верно $x_n \leq y_{1,n} + y_{2,n} + \ldots + y_{k,n} =: z_n$

Нужно доказать, что:
$\uplim\limits_{n \to \infty}{x_n} \leq \uplim\limits_{n \to \infty}{y_{1,n}} + \uplim\limits_{n \to \infty}{y_{2,n}} + \ldots + \uplim\limits_{n \to \infty}{y_{k,n}} =: z_n $ \newline
Потому что верхние пределы --- это и есть Банаховые плотности, о которых идет речь в задаче \newline
1) $\uplim\limits_{n \to \infty}{x_n} \leq \uplim\limits_{n \to \infty}{z_n}$, это верно потому что, $x_n \leq z_n$ \newline
2) $\uplim\limits_{n \to \infty}{z_n} \leq \uplim\limits_{n \to \infty}{y_{1,n}} + \uplim\limits_{n \to \infty}{y_{2,n}} + \ldots + \uplim\limits_{n \to \infty}{y_{k,n}}$, верно потому что, верхний предел суммы не больше суммы верхних пределов \newline

Таким образом, мы доказали, что требуется
\end{proof}

\end{document} % Конец текста.

