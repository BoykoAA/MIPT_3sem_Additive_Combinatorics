
\documentclass[a4paper,12pt]{article}

\usepackage{cmap}					% поиск в PDF
\usepackage[T2A]{fontenc}			% кодировка
\usepackage[utf8]{inputenc}			% кодировка исходного текста
\usepackage[english,russian]{babel}	% локализация и переносы

\documentclass{report}
\usepackage[utf8]{inputenc}
\usepackage[english,russian]{babel}
\usepackage[a4paper,top=1cm,bottom=2cm,left=2cm,right=2cm,marginparwidth=1.75cm]{geometry}
\usepackage{amsfonts}
\usepackage{amsmath}
\usepackage{amsthm}
\usepackage{amssymb}
\usepackage{graphicx}
\usepackage{hyperref}
\usepackage{mathtools}
\usepackage{geometry}
\usepackage{enumitem}
\usepackage{etoolbox}
\usepackage{url}    % package url to prevent horrible linebreaks

\setlength\parindent{0pt}

\numberwithin{equation}{section}
\theoremstyle{plain}
\newtheorem{theorem}{Теорема}[section]
\newtheorem{lemma}[theorem]{Лемма}
\newtheorem{corollary}[theorem]{Corollary}
\newtheorem{proposition}[theorem]{Предложение}
\newtheorem{conjecture}[theorem]{Гипотеза}
\newtheorem{criterion}[theorem]{Criterion}
\newtheorem{algorithm}[theorem]{Algorithm}
\theoremstyle{definition}
\newtheorem{definition}[theorem]{Определение}
\newtheorem{condition}[theorem]{Condition}
\newtheorem{problem}[theorem]{Problem}
\newtheorem{example}[theorem]{Пример}
\newtheorem{exercise}{Exercise}[section]
\newtheorem{obs}{Observation}
\theoremstyle{remark}
\newtheorem{remark}[theorem]{Remark}
\newtheorem{note}[theorem]{Note}
\newtheorem{notation}[theorem]{Notation}
\newtheorem{claim}[theorem]{Claim}
\newtheorem{summary}[theorem]{Резюме}
\newtheorem{acknowledgment}[theorem]{Acknowledgment}
\newtheorem{case[theorem]}{Case}
\newtheorem{conclusion}[theorem]{Conclusion}

\newtheorem*{theorem*}{\bf Теорема}
\newtheorem*{lemma*}{\bf Лемма}
\newtheorem*{definition*}{\bf Определение}
\newtheorem*{example*}{\bf Пример}
\newtheorem*{remark*}{\bf Замечание}
\newtheorem*{hypothesis*}{\bf Гипотеза}


\hypersetup{
  colorlinks   = true,    % Colours links instead of ugly boxes
  urlcolor     = blue,    % Colour for external hyperlinks
  linkcolor    = blue,    % Colour of internal links
  citecolor    = red      % Colour of citations
}

\def\N{\mathbb {N}}
\def\Z{\mathbb {Z}}
\def\F{\mathbb {F}}
\def\d{\delta}
\def\Gr{{\mathbf G}}
\def\E{\mathbb {E}}
\def\eps{\varepsilon}
\def\Sym{{\rm Sym}}
\def\FF{\widehat}
\def\C{\mathbb{C}}

\DeclareMathOperator*\lowlim{\underline{lim}}
\DeclareMathOperator*\uplim{\overline{lim}}

\usepackage{scalerel,stackengine}
\stackMath
\newcommand\reallywidehat[1]{%
\savestack{\tmpbox}{\stretchto{%
  \scaleto{%
    \scalerel*[\widthof{\ensuremath{#1}}]{\kern.1pt\mathchar"0362\kern.1pt}%
    {\rule{0ex}{\textheight}}%WIDTH-LIMITED CIRCUMFLEX
  }{\textheight}% 
}{2.4ex}}%
\stackon[-6.9pt]{#1}{\tmpbox}%
}
\parskip 1ex


%\author{}
%\title{1.1 Наш первый документ}
%\date{\today}

\begin{document} % Конец преамбулы, начало текста.
%\maketitle

Бойко Александр
\newline
\newline
Неделя 1, Задача 7 \newline
Условие:
 Пусть $A, B$ - множества, состоящие из вещественных чисел, притом $|A|, |B| > 1$. Покажите, что
$|A + B| = |A| + |B| - 1$ тогда и только тогда, когда $A$ и $B$ арифметические прогрессии с одинаковой разностью
\begin{proof}
Рассмотрим 2 арифметические прогрессии \newline
$a_1 < a_2 < \ldots a_n$ \newline
$b_1 < b_2 < \ldots < b_m$ \newline
$n,m > 1$ \newline
Пусть $1 \leq i \leq n-1; 1\leq j \leq m-1$

Выпишем суммы членов A+B различными способами по возрастанию: \newline
$a_1+b_1 < a_2+b_1<a_3+b_1<\ldots<a_i+b_1<a_i+b_2<\ldots<a_i+b_j<a_{i+1}+b_j<a_{i+1}+b_{j+1}<a_{i+2}+b_{j+1}<a_{i+3}+b_{j+1}<\ldots<a_n+b_{j+1}<a_n+b_{j+2}<\ldots<a_n+b_b$ \newline
В этой сумме $n+m-1$ чисел и они все лежат в $A+B$ \newline
Заметим, что $|A+B|=n+m-1$ --- по условию \newline
$a_1+b_1 < a_2+b_1<a_3+b_1<\ldots<a_i+b_1<a_i+b_2<\ldots<a_i+b_j<a_{i}+b_{j+1}<a_{i+1}+b_{j+1}<a_{i+2}+b_{j+1}<a_{i+3}+b_{j+1}<\ldots<a_n+b_{j+1}<a_n+b_{j+2}<\ldots<a_n+b_b$ \newline

Последовательности = ВСЕ элементы $A+B$ \newline
$a_i + b_{j+1}=a_{i+1}+b_j\Rightarrow a_{i+1}-a_i=b_{j+1}-b_j$ \newline
Теперь фиксируем $j$ и подставляем в строчку выше и получаем: \newline
$j=1 \Rightarrow b_2-b-1=a_2-a_1=a_3-a_2=a_4-a_3=\ldots$ \newline
Аналогично, фиксируем $i$:
$i=1 \Rightarrow a_2-a_1=b_2-b_1=b_3-b_2=b_4-b_3=\ldots$ \Rightarrow $a$ и $b$ арифметические прогрессии \newline

$a,a+d,a+2d,\ldots,a+(n-1)\cdot d$\newline
$b,b+d,b+2d,\ldots,b+(m-1)\cdot d$\newline

$a+b+kd; 0\leq k\leq n+m-2: n+m-1$ значение, таким образом мы доказали, что требуется




\end{proof}

\end{document} % Конец текста.

