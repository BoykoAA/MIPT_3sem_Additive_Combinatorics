
\documentclass[a4paper,12pt]{article}

\usepackage{cmap}					% поиск в PDF
\usepackage[T2A]{fontenc}			% кодировка
\usepackage[utf8]{inputenc}			% кодировка исходного текста
\usepackage[english,russian]{babel}	% локализация и переносы

\documentclass{report}
\usepackage[utf8]{inputenc}
\usepackage[english,russian]{babel}
\usepackage[a4paper,top=1cm,bottom=2cm,left=2cm,right=2cm,marginparwidth=1.75cm]{geometry}
\usepackage{amsfonts}
\usepackage{amsmath}
\usepackage{amsthm}
\usepackage{amssymb}
\usepackage{graphicx}
\usepackage{hyperref}
\usepackage{mathtools}
\usepackage{geometry}
\usepackage{enumitem}
\usepackage{etoolbox}
\usepackage{url}    % package url to prevent horrible linebreaks

\usepackage{tikz}
\usetikzlibrary{graphs,graphs.standard}


\setlength\parindent{0pt}

\numberwithin{equation}{section}
\theoremstyle{plain}
\newtheorem{theorem}{Теорема}[section]
\newtheorem{lemma}[theorem]{Лемма}
\newtheorem{corollary}[theorem]{Corollary}
\newtheorem{proposition}[theorem]{Предложение}
\newtheorem{conjecture}[theorem]{Гипотеза}
\newtheorem{criterion}[theorem]{Criterion}
\newtheorem{algorithm}[theorem]{Algorithm}
\theoremstyle{definition}
\newtheorem{definition}[theorem]{Определение}
\newtheorem{condition}[theorem]{Condition}
\newtheorem{problem}[theorem]{Problem}
\newtheorem{example}[theorem]{Пример}
\newtheorem{exercise}{Exercise}[section]
\newtheorem{obs}{Observation}
\theoremstyle{remark}
\newtheorem{remark}[theorem]{Remark}
\newtheorem{note}[theorem]{Note}
\newtheorem{notation}[theorem]{Notation}
\newtheorem{claim}[theorem]{Claim}
\newtheorem{summary}[theorem]{Резюме}
\newtheorem{acknowledgment}[theorem]{Acknowledgment}
\newtheorem{case[theorem]}{Case}
\newtheorem{conclusion}[theorem]{Conclusion}

\newtheorem*{theorem*}{\bf Теорема}
\newtheorem*{lemma*}{\bf Лемма}
\newtheorem*{definition*}{\bf Определение}
\newtheorem*{example*}{\bf Пример}
\newtheorem*{remark*}{\bf Замечание}
\newtheorem*{hypothesis*}{\bf Гипотеза}


\hypersetup{
  colorlinks   = true,    % Colours links instead of ugly boxes
  urlcolor     = blue,    % Colour for external hyperlinks
  linkcolor    = blue,    % Colour of internal links
  citecolor    = red      % Colour of citations
}

\def\N{\mathbb {N}}
\def\Z{\mathbb {Z}}
\def\F{\mathbb {F}}
\def\d{\delta}
\def\Gr{{\mathbf G}}
\def\E{\mathbb {E}}
\def\eps{\varepsilon}
\def\Sym{{\rm Sym}}
\def\FF{\widehat}
\def\C{\mathbb{C}}

\DeclareMathOperator*\lowlim{\underline{lim}}
\DeclareMathOperator*\uplim{\overline{lim}}

\usepackage{scalerel,stackengine}
\stackMath
\newcommand\reallywidehat[1]{%
\savestack{\tmpbox}{\stretchto{%
  \scaleto{%
    \scalerel*[\widthof{\ensuremath{#1}}]{\kern.1pt\mathchar"0362\kern.1pt}%
    {\rule{0ex}{\textheight}}%WIDTH-LIMITED CIRCUMFLEX
  }{\textheight}% 
}{2.4ex}}%
\stackon[-6.9pt]{#1}{\tmpbox}%
}
\parskip 1ex


%\author{}
%\title{1.1 Наш первый документ}
%\date{\today}

\begin{document} % Конец преамбулы, начало текста.
%\maketitle

Бойко Александр
\newline
\newline
Неделя 2, Задача 1 \newline
Условие:
Доказать, что множество 
$$
\Gamma_n = \{ x^n ~:~ x\in \F^{*}_p \} \subseteq \F^{*}_p \,.
$$
представляет собой подгруппу по умножению ($n$ является фиксированным). 
\begin{proof} \newline

Для доказательства требуется проверить, что \newline
1) $\Gamma$ - является подмножеством \newline
2) $\Gamma$ - является замкнутым относительно * \newline
3) $\Gamma$ - является замкнутым отностиельно взятия обратного элемента \newline

\newline
1) $x^n$ - это умноение элемента $x$ самого на себя $n$ раз, а следовательно, т.к. наше множество замкнуто относительно * выйти за множество таким образом нельзя $\Rightarrow \Gamma \subset \F_p^*$ \newline
2) Нужно проверить, что $x^n * y^n \in \Gamma$ \newline
$x^n y^n = (xy)^n$ по определению \newline
$(xy)^n \in \Gamma$ \newline
3) Нужно проверить, что если $x^n \in \Gamma$, то и $x^{n^{-1}} \in \Gamma$ \newline
Пусть обратный элемент для $x$ это $y$, т.е. $xy = 1$, заметим, что $y^n \in \Gamma$, тогда $x^n y^n = (xy)^n = 1^n = 1 \Rightarrow$ обратный для $x^n$ это $y^n \in \Gamma$ \newline
\newline
Мы доказали все 3 пунтка и выполнили требование задачи

\end{proof}

Неделя 2, Задача 2 \newline
Условие:
Доказать, что число Рамсея $R(3,3)$ равно 6.
\begin{proof} \newline
Для начала давайте докажем, что $R(3,3) > 5$ это будет обозначать, что в графе $K_5$ при расскраске в 2 цвета, не найдется либо треугольника одного цвета, либо треугольника другого. Приведем в пример вот такую расскраску:


\begin{tikzpicture}[]
    \graph [simple, edges={thick}, clockwise] {
        subgraph C_n [n=5, name=A, radius=2cm]; 

        (A 5)[mark] --[red] (A 4)[mark],
        (A 4)[mark] --[red] (A 3)[mark],
        (A 3)[mark] --[red] (A 2)[mark],
        (A 2)[mark] --[red] (A 1)[mark],
        (A 1)[mark] --[red] (A 5)[mark],
        A 1 -- A 3,
        A 1 -- A 4,
        A 2 -- A 5,
        A 2 -- A 4,
        A 4 -- A 2,
        A 4 -- A 1,
        A 5 -- A 2,
        A 5 -- A 3,
    };

\end{tikzpicture} \newline

Действительно, здесь нет ни красного треугольника, ни черного $\Rightarrow R(3,3) > 5$ \newline
Теперь будем красить $K_6$, если у $K_5$ из каждой вершины выходить 4 ребра, то у $K_6$ по 5 ребер из каждой веришны. Можно как угодно красить 5 ребер в 2 цвета, но мы всегда получим 3 ребра одного цвета, далее будем рассматривать 3 ребра одного цвета.
Тут возникает 2 случая. \newlline
\newpage
1) Между этими ребрами нашлось ребро нужного цвета (из вершины 1 выходят 3 одноцветных ребра, между веришнами 2 и 3 находится красное ребро)

\begin{tikzpicture}[]
    \graph [simple, edges={thick}, clockwise] {
        subgraph C_n [n=4, name=A, radius=2cm]; 

        (A 1)[mark] --[red] (A 2)[mark],
        (A 1)[mark] --[red] (A 3)[mark],
        (A 1)[mark] --[red] (A 4)[mark],
        (A 3)[mark] --[red] (A 2)[mark],
        (A 4)[mark] --[blue] (A 2)[mark],
        (A 3)[mark] --[blue] (A 4)[mark],
        
        
    };

\end{tikzpicture} \newline

В этом случае, нас все устраивает, и красный треугольник нашелся.  \newline
2) Предположим, что ребро не нашлось. Тогда ребра, которые окрашены в другой цвет так же образуют треугольник \newline

\begin{tikzpicture}[]
    \graph [simple, edges={thick}, clockwise] {
        subgraph C_n [n=4, name=A, radius=2cm]; 

        (A 1)[mark] --[red] (A 2)[mark],
        (A 1)[mark] --[red] (A 3)[mark],
        (A 1)[mark] --[red] (A 4)[mark],
        (A 3)[mark] --[blue] (A 2)[mark],
        (A 3)[mark] --[blue] (A 4)[mark],
        (A 4)[mark] --[blue] (A 2)[mark],
        
    };

\end{tikzpicture} \newline

Таким образом, мы можем найти синий треугольник. \newline
В обоих случаях мы можем найти треугольник либо красного, либо синего цвета.



\end{proof}

Неделя 2, Задача 4 \newline
Условие:
Верно ли, что произвольное множество $A\subseteq \N$ имеющее
положительную верхнюю плотность, содержит бесконечную арифметическую прогрессию?
\newline
Нет, не верно.\newline
Приведем пример: \newline
Построим множество по следующему принципу, $\{k(k+1)+1, k(k+1)+2,\ldots,k(k+1)+k+1\}$ \newline
для всех целых неотрицательных k \newline
$C = \{1,3,4,7,8,9,13,14,15,16,21,22,23,24,25 \ldots \} $ \newline
Банахова плотность этого множетсва положительна \newline
$\forall n;  x_n = \frac{|C \cap \{1,2,\ldots,n\}|}{n} \geq \frac{1}{2} > 0$ \newline
Потому что, верхний предел которой берётся в определении банаховой плотности, не меньше $1/2$ для всех n \newline
Пусть есть прогрессия c шагом $d$ \newline
$a,a + d,a + 2d, \ldots$ тогда найдутся $d$ подряд идущих чисел, не лежащих в последовательнсоти, больших $a$. Так мы показали, что множество $C\subseteq \N$ имеющее положительную верхнюю плотность, не содержит бесконечную арифметическую прогрессию



\end{document} % Конец текста.

