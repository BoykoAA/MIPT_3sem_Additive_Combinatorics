
\documentclass[a4paper,12pt]{article}

\usepackage{cmap}					% поиск в PDF
\usepackage[T2A]{fontenc}			% кодировка
\usepackage[utf8]{inputenc}			% кодировка исходного текста
\usepackage[english,russian]{babel}	% локализация и переносы

\documentclass{report}
\usepackage[utf8]{inputenc}
\usepackage[english,russian]{babel}
\usepackage[a4paper,top=1cm,bottom=2cm,left=2cm,right=2cm,marginparwidth=1.75cm]{geometry}
\usepackage{amsfonts}
\usepackage{amsmath}
\usepackage{amsthm}
\usepackage{amssymb}
\usepackage{graphicx}
\usepackage{hyperref}
\usepackage{mathtools}
\usepackage{geometry}
\usepackage{enumitem}
\usepackage[usenames,dvipsnames]{xcolor}
\usepackage{tikz}
\thispagestyle{empty}
\usetikzlibrary{
  graphs,
  graphs.standard
}
\usepackage{etoolbox}
\usepackage{url}    % package url to prevent horrible linebreaks
\usepackage{tikz}
\usetikzlibrary{positioning,chains,fit,shapes,calc}

\usepackage{tikz}
\usetikzlibrary{graphs,graphs.standard}


\setlength\parindent{0pt}

\numberwithin{equation}{section}
\theoremstyle{plain}
\newtheorem{theorem}{Теорема}[section]
\newtheorem{lemma}[theorem]{Лемма}
\newtheorem{corollary}[theorem]{Corollary}
\newtheorem{proposition}[theorem]{Предложение}
\newtheorem{conjecture}[theorem]{Гипотеза}
\newtheorem{criterion}[theorem]{Criterion}
\newtheorem{algorithm}[theorem]{Algorithm}
\theoremstyle{definition}
\newtheorem{definition}[theorem]{Определение}
\newtheorem{condition}[theorem]{Condition}
\newtheorem{problem}[theorem]{Problem}
\newtheorem{example}[theorem]{Пример}
\newtheorem{exercise}{Exercise}[section]
\newtheorem{obs}{Observation}
\theoremstyle{remark}
\newtheorem{remark}[theorem]{Remark}
\newtheorem{note}[theorem]{Note}
\newtheorem{notation}[theorem]{Notation}
\newtheorem{claim}[theorem]{Claim}
\newtheorem{summary}[theorem]{Резюме}
\newtheorem{acknowledgment}[theorem]{Acknowledgment}
\newtheorem{case[theorem]}{Case}
\newtheorem{conclusion}[theorem]{Conclusion}

\newtheorem*{theorem*}{\bf Теорема}
\newtheorem*{lemma*}{\bf Лемма}
\newtheorem*{definition*}{\bf Определение}
\newtheorem*{example*}{\bf Пример}
\newtheorem*{remark*}{\bf Замечание}
\newtheorem*{hypothesis*}{\bf Гипотеза}


\hypersetup{
  colorlinks   = true,    % Colours links instead of ugly boxes
  urlcolor     = blue,    % Colour for external hyperlinks
  linkcolor    = blue,    % Colour of internal links
  citecolor    = red      % Colour of citations
}

\def\N{\mathbb {N}}
\def\Z{\mathbb {Z}}
\def\F{\mathbb {F}}
\def\d{\delta}
\def\Gr{{\mathbf G}}
\def\E{\mathbb {E}}
\def\eps{\varepsilon}
\def\Sym{{\rm Sym}}
\def\FF{\widehat}
\def\C{\mathbb{C}}

\DeclareMathOperator*\lowlim{\underline{lim}}
\DeclareMathOperator*\uplim{\overline{lim}}

\usepackage{scalerel,stackengine}
\stackMath
\newcommand\reallywidehat[1]{%
\savestack{\tmpbox}{\stretchto{%
  \scaleto{%
    \scalerel*[\widthof{\ensuremath{#1}}]{\kern.1pt\mathchar"0362\kern.1pt}%
    {\rule{0ex}{\textheight}}%WIDTH-LIMITED CIRCUMFLEX
  }{\textheight}% 
}{2.4ex}}%
\stackon[-6.9pt]{#1}{\tmpbox}%
}
\parskip 1ex


%\author{}
%\title{1.1 Наш первый документ}
%\date{\today}

\begin{document} % Конец преамбулы, начало текста.
%\maketitle

Бойко Александр
\newline
\newline
Неделя 5, Задача 1 \newline
Условие. Построить множество $A \subseteq \F_2^n$
такое, что $|A| = 2^{n-1}$ и $A$ не содержит
нетривиальных решений уравнения $x+y+z = 0$.
Этот пример показывает, что если задача неафинна, то
могут существовать очень большие множества без
решений соответствующего линейного уравне \newline
Решение: \newline
Рассмотрим множетсва как векторы их чисел и возьмем только те векторы, в которых нечетное кол-во единиц и $2^{n-1}$ штука, докажем это
\begin{proof}
$$C_n^0 + C_n^2 + C_n^4 + \ldots = C_n^1 + C_n^3 + C_n^5 + \ldots$$
$$(a+b)^n = \sum_{k=0}^n a^k \cdot b^{n-k} \cdot C_n^k$$
$a=1, b=-1$
$$0^n = (-1)^0 \cdot 1^n \cdot C_n^0 + (-1)^1 \cdot 1^{n-1} \cdot C_n^1 +(-1)^2 \cdot 1^{n-2} \cdot C_n^2 + \ldots = C_n^0 + C_n^1 + C_n^2 - C_n^3 + \ldots$$
Следовательно четных и нечетных поровну \Rightarrow нечетных $2^{n-1}$
\end{proof}
Введем вспомогательный вектор $$e = (1,1,1, \ldots , 1)$$
Тогда, $\forall a \in A: (a,e) = 1$ \newline
Пусть $x,y,z \in A; x+y+z=0$ \newline
$$0 = (0,e) = (x+y+z, e) = (x,e) + (y,e) + (z,e) = 1+1+1 = 1$$
Получили противоречие \newline
Задача 2 \newline
Условие.Найти множество $A \subseteq (\Z/3\Z)^n$ такое, что $|A| \ge 2^n$ и $A$ не содержит нетривиальных решений уравнения $x+y = 2$ \newline
Решение. \newline
Рассмотрим множество $A$ в котором на каждой позиции вектора 0 или 1, в нем $2^N$ элементов (на каждую позицию 2 варианта, а позиций $N$). \newline
Проверим, что нет нетривиальных решений. \newline
Допустим есть какое-то нетривиальных нетривиальное. \newline
Докажем, что $x, y$ различны. Если бы $x, y$ совпадают, то и $z$ совпадал бы с ними, тогда было бы тривиальным. \newline
Они отличаются, тогда когда найдется позиция вектора, где на месте одного единица, а на месте другого ноль, тогда $x+y$ точно равно $1$, тогда $z=2$, а это противоречие.


\end{document} % Конец текста.

