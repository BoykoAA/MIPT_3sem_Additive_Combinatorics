
\documentclass[a4paper,12pt]{article}

\usepackage{cmap}					% поиск в PDF
\usepackage[T2A]{fontenc}			% кодировка
\usepackage[utf8]{inputenc}			% кодировка исходного текста
\usepackage[english,russian]{babel}	% локализация и переносы

\documentclass{report}
\usepackage[utf8]{inputenc}
\usepackage[english,russian]{babel}
\usepackage[a4paper,top=1cm,bottom=2cm,left=2cm,right=2cm,marginparwidth=1.75cm]{geometry}
\usepackage{amsfonts}
\usepackage{amsmath}
\usepackage{amsthm}
\usepackage{amssymb}
\usepackage{graphicx}
\usepackage{hyperref}
\usepackage{mathtools}
\usepackage{geometry}
\usepackage{enumitem}
\usepackage[usenames,dvipsnames]{xcolor}
\usepackage{tikz}
\thispagestyle{empty}
\usetikzlibrary{
  graphs,
  graphs.standard
}
\usepackage{etoolbox}
\usepackage{url}    % package url to prevent horrible linebreaks
\usepackage{tikz}
\usetikzlibrary{positioning,chains,fit,shapes,calc}

\usepackage{tikz}
\usetikzlibrary{graphs,graphs.standard}


\setlength\parindent{0pt}

\numberwithin{equation}{section}
\theoremstyle{plain}
\newtheorem{theorem}{Теорема}[section]
\newtheorem{lemma}[theorem]{Лемма}
\newtheorem{corollary}[theorem]{Corollary}
\newtheorem{proposition}[theorem]{Предложение}
\newtheorem{conjecture}[theorem]{Гипотеза}
\newtheorem{criterion}[theorem]{Criterion}
\newtheorem{algorithm}[theorem]{Algorithm}
\theoremstyle{definition}
\newtheorem{definition}[theorem]{Определение}
\newtheorem{condition}[theorem]{Condition}
\newtheorem{problem}[theorem]{Problem}
\newtheorem{example}[theorem]{Пример}
\newtheorem{exercise}{Exercise}[section]
\newtheorem{obs}{Observation}
\theoremstyle{remark}
\newtheorem{remark}[theorem]{Remark}
\newtheorem{note}[theorem]{Note}
\newtheorem{notation}[theorem]{Notation}
\newtheorem{claim}[theorem]{Claim}
\newtheorem{summary}[theorem]{Резюме}
\newtheorem{acknowledgment}[theorem]{Acknowledgment}
\newtheorem{case[theorem]}{Case}
\newtheorem{conclusion}[theorem]{Conclusion}

\newtheorem*{theorem*}{\bf Теорема}
\newtheorem*{lemma*}{\bf Лемма}
\newtheorem*{definition*}{\bf Определение}
\newtheorem*{example*}{\bf Пример}
\newtheorem*{remark*}{\bf Замечание}
\newtheorem*{hypothesis*}{\bf Гипотеза}


\hypersetup{
  colorlinks   = true,    % Colours links instead of ugly boxes
  urlcolor     = blue,    % Colour for external hyperlinks
  linkcolor    = blue,    % Colour of internal links
  citecolor    = red      % Colour of citations
}

\def\N{\mathbb {N}}
\def\Z{\mathbb {Z}}
\def\F{\mathbb {F}}
\def\d{\delta}
\def\Gr{{\mathbf G}}
\def\E{\mathbb {E}}
\def\eps{\varepsilon}
\def\Sym{{\rm Sym}}
\def\FF{\widehat}
\def\C{\mathbb{C}}

\DeclareMathOperator*\lowlim{\underline{lim}}
\DeclareMathOperator*\uplim{\overline{lim}}

\usepackage{scalerel,stackengine}
\stackMath
\newcommand\reallywidehat[1]{%
\savestack{\tmpbox}{\stretchto{%
  \scaleto{%
    \scalerel*[\widthof{\ensuremath{#1}}]{\kern.1pt\mathchar"0362\kern.1pt}%
    {\rule{0ex}{\textheight}}%WIDTH-LIMITED CIRCUMFLEX
  }{\textheight}% 
}{2.4ex}}%
\stackon[-6.9pt]{#1}{\tmpbox}%
}
\parskip 1ex


%\author{}
%\title{1.1 Наш первый документ}
%\date{\today}

\begin{document} % Конец преамбулы, начало текста.
%\maketitle

Бойко Александр
\newline
\newline
Неделя 6, Задача 1 \newline
Условие. Найти константу удвоения произвольного базиса в $\F_p^n$. \textit{Напоминание:}  константа удвоения множества $X$ есть $k[X] := \frac{|X + X|}{|X|}$. \newline
Пусть $y$ -- базис, тогда в $y$ $n$ элементов $\F_p^n$ имеет размерность $n$. \newline
Поскольку каждый элемент $\F_p^n$ может быть выражен одним только способом, через элементы базиса \Rightarrow выражение вида $y_1 + y_2$ ($y_1, y_2 \in y$) попарно различны. Соотственно $|y+y|$ -- количество элементов при сложении $$|y+y| = n + C_n^2 = \frac{n(n+1)}{2}$$
$$k[y] = \frac{\frac{n(n+1)}{2}}{n} = \frac{n+1}{2}$$
Однако при $p=2$ векторы имеющие координаты $0,0, \ldots 2, \ldots, 0$ равны нулевому вектору, поэтому все элементы $y_1+y_1$, где $y_1 \in y$ одинаковы \Rightarrow в случае $p=2$ ответ будет 

$$k[y] := \frac{n+C_n^2 - (n-1)}{n}$$ \newline
\newline

Задача 2 \newline
Условие. Доказать, что если $P$ --- обобщенная AP и $\varphi$ --- гомоморфизм порядка два, то $\varphi (P)$ --- обобщенная AP той же размерности.
\begin{proof}
Докажем, что если $P$ обобщенная $AP$ и $\varphi$ -- гомоморфизм порядка два, то $\varphi(P)$ -- обобщенная $AP$ той же размерности. \newline
Рассмотрим арифметическую прогрессию $\{a, a+b, \ldots, a + (k-1)d\}$. По определению гомоморфизма Фреймана

$$(a+d) + 0 = a + d \Rightarrow \varphi(a+d) + \varphi(0) = \varphi(a) + \varphi(d)$$.
Отсюда 
$$\varphi(a+d)=\varphi(a) + \vatphi(d) - \varphi(0)$$.
Далее
$$(a+2d) + 0 = (a+d)+d \Rightarrow \varphi(a+2d) + \varphi(0) = \varphi(a+d) + \varphi(d)$$

Значит, $\varphi(a+2d) = \varphi(a) + 2(\varphi(d) - \varphi(0))$ и т.д. То есть образ прогрессии -- прогрессия. Для многомерной прогрессии тоже самое.

\end{proof}

\end{document} % Конец текста.

